%NOME: Francisco Rosa Dias de Miranda
%CÓDIGO: 4402962
%CURSO: Bacharelado em Estatística
%DISCIPLINA: Laboratório de Computação e Simulação

\documentclass[a4paper]{article}
\usepackage[utf8]{inputenc}
\usepackage[T1]{fontenc}
\usepackage{lmodern}
\usepackage{hyperref}
\usepackage{mathtools}

\begin{document}

\title{MAP2212 \\ Laboratório de Computação e Simulação \\ EP5}
\author{Francisco Rosa Dias de Miranda}

\maketitle

\section{Sobre o programa}

O arquivo \textit{ep5.r} foi desenvolvido em \emph{R}, usando a
bibliotecas \emph{nloptr}.

\section{Full Bayesian Significance Test}

O \textit{FBST} é um teste que pode ser realizado para, usando apenas a
probabilidade posterior e valores observados, verificar se uma certa hipótese é
válida ou não. O teste segue o princípio de Bayes e é ``Completo'' no sentido
que apenas a distribuição \textit{a posteriori} é usada, e não precisamos de
informação sobre a distribuição além disso.

Neste EP, realizamos o teste descrito no enunciado, que consiste, em poucas
palavras, em verificar se um certo fabricante de painéis de LED está sendo
honesto quanto ao tempo médio de falha de seus produtos.

O teste consiste em criar um modelo estatístico para representar a nossa
variável de interesse (nesse caso, tempo de vida dos painéis de LED), e
descobrir, a partir de um conjunto de informações observadas, possíveis
par\^ametros para essa variável e ver se eles condizem com a informação
publicada pelo fabricante.

Para realizar o teste, precisamos achar o conjunto de par\^ametros \(\theta^*\)
que tem probabilidade máxima, dada a restrição de nossa hipótese nula e os
valores observados. Em outras palavras, achamos o valor mais provável para os
par\^ametros dado o valor de \(\rho\) do fabricante. Para implementar isso,
usamos a função \textit{minimize} da biblioteca \emph{SciPy}.

Depois disso, precisamos verificar quantos valores dentro do espaço de
possíveis par\^ametros são mais prováveis do que os valores que encontramos. Ou
seja, achamos a quantidade de pontos mais prováveis do que o mais provável
restrito à hipótese nula. A partir desse número, podemos dizer, com uma certa
confiança, se o valor dado pelo fabricante está correto ou não. Quanto maior a
probabilidade, maior a chance de que o fabricante mentiu, pois a chance de que
o valor real não satisfaz a hipótese nula é maior. Para implementar essa parte,
implementamos o algoritmo \textit{Metropolis-Hastings} de \textit{Markov Chain
Monte Carlo}.

\section{Conclusão}

As estimativas de máxima verossimilhança para $\alpha$, $\beta$ e $\gamma$ obtidas foram, respectivamente, $<- 1.25, <- 3.28, <- 3.54$.

O ponto de máximo encontrado foi 0.

O teste completamente Bayesiano (FBST) é muito útil para verificar a validade
de uma hipótese estatística, apesar de ser difícil de implementar ele por
completo e sem erros. O uso de bibliotecas externas facilita bastante o
desenvolvimento.

A sua eficácia, porém, depende bastante do Kernel de Transição escolhido.
Apesar disso, as vezes é a única opção factível em problemas com um alto número
de dimensões.

\end{document}

