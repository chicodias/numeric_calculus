%NOME: Francisco Rosa Dias de Miranda
%CÓDIGO: 4402962
%CURSO: Bacharelado em Estatística
%DISCIPLINA: Laboratório de Computação e Simulação

\documentclass[a4paper]{article}
\usepackage[portuguese]{babel}
\usepackage[utf8]{inputenc}
\usepackage{amsmath}
\usepackage{amssymb}
\usepackage{graphicx}
\usepackage{url}
\usepackage[colorinlistoftodos]{todonotes}

\title{Integrais pelo Metodo de Monte Carlo}

\author{Francisco Rosa Dias de Miranda- Nº USP 4402962}

\date{\today}

\begin{document}
\maketitle\

\section{Descrição do EP2:}

Este exercício programa foi projetado em R e teve como objetivo calcular numericamente $\int_{0}^{1}f(t)dt$, sendo \[f(t)=e^{(-0.2506(1+t) (0.5) (1+\cos0.6440) t)}\gamma(t)\] com $\gamma(t) \approx 1$.
Como parâmetros deste programa, temos a precisão desejada, que foi estipulada em 3 casas depois da vírgula. A saída é impressa na tela com os valores da integral por cada método, sua respectiva precisão e o número de pontos utilizando os métodos:
\begin{itemize}
\item Monte Carlo Cru
\item Hit-Or-Miss
\item Importance Sampling
\item Função quadrática como variável de controle
\end{itemize}
A seguir, descreveremos brevemente cada método:

\subsection{Monte Carlo Cru}
  Considerado o mais simples, este método consiste em gerar números aleatórios que possuam distribuição uniforme dentro do intervalo $[a,b]$ do domínio da função $f(t)$ que queremos integrar. A estimativa da integral é dada por:
  
\[\int_0^1 f(t)dt\approx E[f(t)]=\dfrac{1}{n}\sum_{k=1}^nf(t_k)\]

\subsection{Hit-Or-Miss} 

  Para estimar a integral através do método hit-or-miss, primeiramente determinamos um retângulo de base $(b-a)$ e altura $h = 1$, pois em nosso caso $f(t)<1, \forall t \in [0,1]$. Geramos então N pares de números aleatórios $(t_i,y_i)$, onde $t_i\sim U([a,b])$, $y_i\sim U([0,h])$ e temos a integral estimada por:
\[ \int_0^1f(t)dt\approx A\dfrac{n}{N},\]
onde $A=(a-b)h$ é a área do retângulo que contém o gráfico de $f$, $n$ é o número de pontos em que $y_i<f(t_i)$ e $N$ é o número total de pontos gerados.

\subsection{Importance Sampling}

Neste método, iremos gerar números aleatórios em regiões onde $f(t)$ é maior, para aumentar a eficiência. Usamos uma distribuição $g(t)$ de tal forma que $\dfrac{f(t)}{g(t)}\approx C$, com C constante. Temos então um estimador para $\int f(t)dt$:
\[\int f(t)dt=\int \dfrac{f(t)}{g(t)}g(t)dt=\int \dfrac{f(t)}{g(t)}dg\approx\dfrac{1}{n}\sum_{k=1}^{n}\dfrac{f(x_k)}{g(x_k)},\]
com $x_k\sim g$. Neste EP utilizamos a distribuição beta com parâmetros $a=1$ e $b=1,5$ como função $g$.


\subsection{Monte Carlo com função quadrática como variável de controle}

Neste caso, a variável de controle será uma função $h(t)$ "próxima" de $f(t)$ e facilmente integrável analiticamente. Fazendo
\[\int_0^1f(t)dt=\int_0^1(f(t)+h(t)-h(t))dt=\int_0^1h(t)dt+\int_0^1(f(t)-h(t))dt\]

teremos $\int_0^1(f(t)-h(t))dt\approx 0$, pois $f(t)\approx h(t)$. Como $\int_0^1h(t)dt$ pode ser calculado de maneira exata analiticamente, teremos um estimador para a integral dado por
\[\int_0^1f(t)dt\approx \int_0^1g(t)dt+\dfrac{1}{n}\sum_{k=1}^n(f(t_k)-h(t_k))\]
No caso de $h(t)\approx f(t)$, teremos $\dfrac{1}{n}\sum_{k=1}^n(f(t_k)-h(t_k))\approx 0$, o que irá diminuir a variância deste método em relação ao método de Monte Carlo Cru, tornando-o mais eficiente. No caso deste EP, foi utilizada uma função quadrática $h(t)=at^2+bt+c$. A determinação das constantes $a$, $b$ e $c$ é discutida adiante. 

\section{Número de pontos e critério de parada}

	Neste EP, o critério de parada utilizado foi o desvio padrão da média $\sigma_n$. Para um valor $\epsilon$ desejado, o algoritmo executa o cálculo da integral, inicialmente com 1000 pontos. Se $\sigma_n>\epsilon I$, onde $I$ é o valor estimado para $\int_0^1f(t)dt$, acrescentamos mais 1000 pontos e repetimos o processo sucessivamente até que $\sigma_n<\epsilon I$. Desta forma obtemos um erro relativo da mesma ordem de grandeza de $\epsilon$.

\section{Determinação dos coeficientes $a$, $b$ e $c$ utilizados no método de Monte Carlo por Variável de Controle}

	Sabemos que qualquer função quadrática em $\mathbb{R}$ é unicamente determinada por seu valor em 3 pontos. Para encontrar uma variável de controle que se aproxime de $f$, precisamos de um polinômio de grau $2$ que interpole $f$ o mais proximo possível. Como queremos interpolar $f$ no intervalo $[0,1]$, utilizamos os pontos $1/6$, $1/2$, $5/6$, de forma a garantir que nenhum ponto do domínio fique a uma distância maior do que $1/6$ de um ponto onde a aproximação é exata. Logo, queremos obter o polinômio $ax^2+bx+c$ que passe pelos pontos $(\frac{1}{6}, f(\frac{1}{6})), (\frac{1}{2},f(\frac{1}{2}))$ e $(\frac{5}{6},f(\frac{5}{6}))$. Para tal, basta resolver o sistema linear a seguir para a, b e c :
  
$$\left\{ \begin{array}{c}
a (1/6)^2 + b (1/6) + c = f(1/6)\\
a (1/2)^2 + b (1/2) + c = f(1/2)\\
a (5/6)^2 + b (5/6) + c = f(5/6)\\
\end{array}
\right.$$

Ao resolvê-lo, obtemos  $a\approx 0.05770457\textrm{, }b\approx -0.19983003\textrm{ e }c\approx 0.77882512$. Colocando $g(t)$ em forma de uma soma de monômios, temos 
\[g(t)\approx 0.05770457t^2 -0.19983003t + 0.77882512 \]

\section{Testes numéricos e discussão:}{
A saída do programa gerou resultados conforme esperávamos. O número de pontos foi menor conforme o grau de refinamento do método, sendo que o de Variável de Controle foi o mais rápido apesar de necessitar de bem menos amostras, pois a escolha de $g(t)$ aproximou-se de $f(t)$ em 6 casas decimais em $[a,b]$. O método mais demorado foi o Importance Sampling, possivelmente devido aos cálculos envolvendo a função beta. Um exemplo dos resultados obtidos é exibido na tabela 1.

\begin{table}[]
\centering
\begin{tabular}{lllll}
\multicolumn{1}{c}{Método} & \multicolumn{1}{c}{$n$} & \multicolumn{1}{c}{$I$} & \multicolumn{1}{c}{$\epsilon$} &  \\
Crude & 602000 & 0.6981666 & 8.998307e-04 &  \\
Hit or Miss & 261000 & 0.6991188 & 8.977463e-04 &  \\
Importance Sampling & 137000 & 0.6974152 & 8.983215e-04 &  \\
Variável de Controle & 2000 & 0.6981370 & 2.783546e-06 & 
\end{tabular}
\caption{Exemplo de resultados obtidos usando o programa do EP2}
\label{my-label}
\end{table}
}
\end{document}